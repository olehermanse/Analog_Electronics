\subsection{Physical properties}
When a length of wire is formed into a coil, it becomes an inductor. Current through the coil produces an electromagnetic field. The magnetic lines of force around each loop in the winding of the coil effectivly add to the lines of force around the adjoining loops, forming a stronger magnetic field within and around the coil. The number of turns of wire, the length and the cross-sectional area of the core are factors in setting the value of inductance. This relationship is as follows : 
\[L = \frac{N^2 \mu A}{l}\]

L is the Inductance in henries (H), N is the number of turns, $\mu$ is the permeability in henries per meter (H/m), A is the cross-sectional area in $m^2$, and l is the core length in meters.

\subsection{Induced voltage}
Voltage produced as a result of a changing magnetic field. This property is called self-inductance but is usually referred to as simply inductance, with symbol L.

\subsection{Inductive reactance}
Inductive reactance is the opposition to sinusoidal current in an inductor. The symbol is $X_L$, and its unit is ohm. Less inductance, the more current. The more inductance, the less current. Also current decreases when the frequency increases, and the other way around, when the frequency decreases the current increases.

\subsection{Series and parallel inductors}
When inductors are connected in \emph{series}, the total inductance is the sum of the individual inductances. The formula is
\[L_T = L_1 + L_2 + ... + L_n\]
It is the same formula for reactance.

\noindent When inductors are connected in \emph{paralell}, the total inductance is less than the samllest inductance. The formula is 
\[\frac{1}{L_T} = \frac{1}{L_1} +\frac{1}{L_2} + ... + \frac{1}{L_n}\]
There is a small change in the formula fo reactance in paralell, then the formula is 
\[X_{L(tot)} = \frac{1}{\frac{1}{L_1} +\frac{1}{L_2} + ... + \frac{1}{L_n}}\]

\subsection{Practical Inductors}
A practical inductor leaks reactive inductance, resistance and capacitance. 


\subsection{RL circuits}
The time constant for a series RL circuit is the inductance divided by the resistance. In an RL circuit, the increasing or decreasing voltage and current in an inductor make an approximately 63\% change during each time-constant interval. 
