\subsection{Fundamentals}
A circuit consists of \emph{nodes}, \emph{paths}, \emph{loops}, \emph{branches}
and \emph{components}. A \emph{component} can be a resistor, capacitor, 
battery, LED etc. \emph{Components} are connected by wires, a \emph{node} 
consists of all the wires that are connected together without any (significant) 
impedance. There is one \emph{node} on each pin of a \emph{component}. Pins or 
wires are in the same \emph{node} if they are directly connected, and thus 
have the same voltage. A \emph{path} is a road (through \emph{components}) 
where a current can flow. A \emph{loop} is a closed \emph{path} that connects 
to itself. A closed circuit has one or more \emph{loops} and \emph{paths}.

\subsection{Series circuits}
When connecting resistors in series we sum their resistance values to get the 
total:
\begin{equation}
    R_{TOT} = R_1 + R_2 + ... + R_n
\end{equation}
The same is true for non-resistive impedances.
\subsection{Parallel circuits}
When connecting resistors in parallel we sum their conductances to find total 
conductance:
\begin{equation}
    G_{TOT} = G_1 + G_2 + ... + G_n
\end{equation}
This makes sense; connecting two wires or equal resistors will double the 
conductance. We can easily find the resistance of two parallell resistors:
\begin{equation}
    R_{TOT} = \frac{1}{G_{TOT}} = \frac{1}{G_1 + G_2} = 
    \frac{1}{\frac{1}{R_1} + \frac{1}{R_2}}
\end{equation}
Or the common \emph{shortcut}:
\begin{equation}
    R_{TOT} = \frac{R_1R_2}{R_1+R_2}
\end{equation}
\subsection{Kirchhoffs Voltage Law}
The sum of all voltages going around a loop in a circuit is always 0.
\begin{equation}
    V_1 + V_2 + ... + V_n = 0
\end{equation}
This is because when we end up at the same node, we must also end up at the 
original voltage level.
\subsection{Kirchhoffs Current Law}
The sum of currents into/out of a node is always 0.
\begin{equation}
    I_1 + I_2 + ... + I_n = 0
\end{equation}
This is because there has to come as much charge out of as into a node. Beware that sign(direction) of current is important!
\subsection{Voltage divider}
A voltage divider takes an input voltage and produces a lower output voltage, 
that is a fraction of the input.
\begin{equation}
    V_x = V_S \frac{R_x}{R_{TOT}}
\end{equation}
Two or more resistors are used to split the voltage into different levels. The 
relative values of the resistors decide the voltages. For example: if one 
resistor is twice as big as the other, the voltage drop over it will be twice 
as big as the voltage drop over the other. A potentiometer is often used to 
create an adjustable output voltage.
\subsubsection{Load}
It is important to note that when connecting the voltage divider to a load the 
output voltage changes(!). Adding a load to the output will decrease the 
resistance down to ground and lower the voltage. A voltage divider with lower 
resistor values is better at supplying a lot of current to a load without 
changing output voltage (much).

\subsection{Current divider}
Current dividers are very similar but not as common as voltage dividers. 
When connecting resistors in parallel the current is divided among them. Recall 
the update ohm's law in \vref{eq:ohmconductance}:
\begin{equation*}
    I = V*G
\end{equation*}
Thus the current through one of the parallel resistors is:
\begin{equation*}
    I_X = V_{IN}*G_X
\end{equation*}
Total conductance is simply:
\begin{equation*}
    G_{TOT} = G_1 + G_2 + ... + G_n
\end{equation*}
And total current through all the resistors is:
\begin{equation*}
    I_{TOT} = V_{IN}*G_{TOT}
\end{equation*}
The more standard current divider formula is obtained by specifying input
current instead of voltage:
\begin{equation}
    I_X = (V_{IN})*G_X = (R_{TOT}*I_{IN})*G_X = \frac{G_X}{G_{TOT}}*I_{IN}
\end{equation}
Or in terms of resistance:
\begin{equation}\label{eq:currentdivider}
    I_X = \frac{R_{TOT}}{R_X} * I_{IN}
\end{equation}
It is crucial to remember that this is a parallel circuit, so:
\begin{equation*}
    R_X \geq R_{TOT}
\end{equation*}