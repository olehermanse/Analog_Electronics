\subsection{RLC characteristics}
$X_L$ and $X_C$ have opposing effects in an RLC circuit. In a series RLC circuit, the larger reactance determines the net reactance of the circuit. In a parallel RLC cicuit, the smaller reactance determines the net reactance of the circuit. 

\subsection{Time domain}


\subsection{Frequency domain}
In electric circuits, the variation in the output voltage (or current) over a specified range of frequencies.

\subsection{Harmonics}
A repetitive nonsinusoidal waveform contains sinusoidal waveforms with a fundamental frequency and harmonic frequencies. The fundamental frequency is the repetition rate of the waveform, and the harmonics are higher-frequency sine waves that are multiples of the fundamental. 
\begin{itemize}
    \item Odd Harmonics are frequencies that are odd multiples of the fundamental frequency of a waveform.
    
    \item Even Harmonics are frequencies that are even multiples of the fundamental frequency. 
\end{itemize}
Composite waveform is any variation from a pure sine wave which produces harmonics. 

\subsection{Fourier series and Fourier transform}


\subsection{Low-Pass filter}
A low-pass filter is used to only let the low frequencies pass through the circuit. 

\subsection{High-Pass filter}
A high-pass filter is used to only the the high frequencies pass through the circuit. 

\subsection{Band-pass filter}
A \emph{band-pass filter} allows signals at the resonant frequency and at frequencies within a certain band (or range) extending below and above the resonant value to pass from input to output without a significant reduction in amplitude. Signals at frequencies lying outside this specified band (called the \emph{passband}) are reduced in amplitude to below a certain level and are considered to be rejected by the filter.
The formula for calculating the bandwidth is 
\[BW = f_2 - f_1\]

\subsection{Band-stop filter}
The \emph{band-stop filter} rejects signals with frequencies between the lower and upper cutoff frequencies and passes those signals with frequencies below and above the cutoff values. The range of frequencies between the lower and upper cutoff points is called the stopband. This type of filter is also referred to as a \emph{band-elimination filter}, \emph{band-reject filter}, or a \emph{notch filter}. 
